%%________________________________________________________________________
%% LEIM | PROJETO
%% 2022 / 2013 / 2012
%% Modelo para relat�rio
%% v04: altera��o ADEETC para DEETC; outros ajustes
%% v03: corre��o de gralhas
%% v02: inclui anexo sobre utiliza��o sistema controlo de vers�es
%% v01: original
%% PTS / MAR.2022 / MAI.2013 / 23.MAI.2012 (constru�do)
%%________________________________________________________________________




%%________________________________________________________________________
\chapter{Introdu��o}
\label{ch:introducao}
%%________________________________________________________________________

Este trabalho visa criar um jogo em Unity \cite{Unity}, utilizando os conhecimentos adquiridos ao longo do semestre, nomeadamente gera��o procedimental, aprendizagem autom�tica e intera��o Pessoa-M�quina.

O projeto consiste na cria��o de um mundo Minecraft \cite{Minecraft}, onde agentes s�o treinados a desempenhar determinadas tarefas e o jogador consegue interagir com o ambiente, modificando o curso dos acontecimentos.

Este documento est� organizado da seguinte forma:

TODOTODOTODO esperar final e/ou confirma��o led TODOTODOTODOTODO


%%________________________________________________________________________
\chapter{Ambiente}
\label{ch:ambiente}
%%________________________________________________________________________

Neste cap�tulo ser� ilustrado a forma como o terreno � criado, os tipos de blocos existentes no jogo e o aspeto com a \aspas{Skybox}.

%%________________________________________________________________________
\section{Terreno}
\label{sec:terreno}
%%________________________________________________________________________

forma como o terreno � criado

%%________________________________________________________________________
\section{Blocos}
\label{sec:blocos}
%%________________________________________________________________________

tipos de blocos (grass, dirt, stone etc)

%%________________________________________________________________________
\section{Skybox}
\label{sec:skybox}
%%________________________________________________________________________

mostrar aspeto final da skybox

%%________________________________________________________________________
\chapter{Agentes}
\label{ch:agentes}
%%________________________________________________________________________

Aqui s�o identificados os agentes aut�nomos do projeto e a forma como foram treinados. 

%%________________________________________________________________________
\section{Sistema Natural}
\label{sec:sistemanatural}
%%________________________________________________________________________

explicar que existem lobos que comem ovelhas etc etc

%%________________________________________________________________________
\section{ML-Agents}
\label{sec:mlagents}
%%________________________________________________________________________

resumir os ml agents, a rede neuronal e explicar os passos como foram treinados

%%________________________________________________________________________
\section{Ficheiro de Configura��o}
\label{sec:ficheiroconfiguracao}
%%________________________________________________________________________

explicar o ficheiro usado

%%________________________________________________________________________
\chapter{Interface}
\label{ch:interface}
%%________________________________________________________________________

Este cap�tulo ilustra a interface do jogo.

%%________________________________________________________________________
\section{Menu}
\label{sec:menu}
%%________________________________________________________________________

mostrar menu(s)

%%________________________________________________________________________
\section{Lifebar}
\label{sec:lifebar}
%%________________________________________________________________________

mostrar lifebars dos animais e/ou outras (acho que queres para o jogador tamb�m)

%%________________________________________________________________________
\chapter{Intera��o}
\label{ch:interacao}
%%________________________________________________________________________

Este cap�tulo trata a forma como o jogador interage com o ambiente, nomeadamente recorrendo ao teclado e microfone.

%%________________________________________________________________________
\section{Teclado}
\label{sec:teclado}
%%________________________________________________________________________

criar e destruir cubos

spawnar comida

%%________________________________________________________________________
\section{Microfone}
\label{sec:microfone}
%%________________________________________________________________________

ainda n�o temos definido, mas ya � obrigat�rio

%%________________________________________________________________________
\chapter{Conclus�o}
\label{ch:conclusao}
%%________________________________________________________________________
